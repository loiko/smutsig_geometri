\documentclass[10pt,a4paper]{article}
\usepackage[utf8]{inputenc}
\usepackage[T1]{fontenc}
\usepackage{graphicx}
\usepackage{amsmath}
\usepackage{amssymb}
\usepackage{amsthm}
\usepackage[swedish]{babel}
\usepackage[colorlinks]{hyperref}
\usepackage{url}
%\usepackage{qtree}
\usepackage{color}
\usepackage{todonotes}

% commutative diagrams
\usepackage[all,cmtip]{xy}
%\usepackage{alg}

%\usepackage{youngtab}

% Add this one at the last possible moment,
% don't forget -shell-escape (one '-'),
% never use anything non-ascii
% and just delete everything and start again
% if you get strange errors.
% Oh, and did I mention, you have to have python
% and pygments installed!
% On the other hand, the results are worth it!
% \usepackage{texments}
% \usestyle{default}

\usepackage{breqn}

\newtheorem{Theorem}{Theorem}
\newtheorem{Uppgift}{Uppgift}
\newtheorem{Lemma}{Lemma}
\newtheorem{Lösning}{Lösning}
\newtheorem{Solution}{Solution}
\newtheorem{Definition}{Definition}
\newtheorem{Example}{Example}
\newtheorem{Sats}{Sats}
\newtheorem{Exempel}{Exempel}
\newtheorem{Proposition}{Proposition}
\newtheorem{Corollary}{Corollary}
\newtheorem{Exercise}{Exercise}
\newtheorem{Claim}{Claim}

\newcommand{\bigRedBox}[1]
{
  \fbox{
    \parbox{300px}{
      \color{red}
      #1
    }
  }
}

\newcommand{\redBox}[1]
{
  \fbox{
    \color{red}
    #1
  }
}


\newcommand{\smallLine}
{\begin{center}
    \line(1,0){100}
  \end{center}
}
\newcommand{\myLine}
{\begin{center}\line(1,0){200}\end{center}}

\newcommand{\myInput}[1]
{
  \vspace{1cm}
  \begin{center}
    \textbf{File: #1}\\  
    \line(1,0){350}
  \end{center}
  \input{#1}
}

%\newcommand{\dd}{\text{d}}
\newcommand{\Lla}{\Longleftarrow}
%\newcommand{\llra}{\longleftrightarrow}
%\newcommand{\lra}{\longrightarrow}
%\newcommand{\Lra}{\Longrightarrow}


%\addtolength{\textwidth}{1in}
%\addtolength{\hoffset}{-0.5in}
%\addtolength{\textheight}{1in}
%\addtolength{\voffset}{-1in}

\newcommand{\twoMat}[4]{
  \ensuremath{
    \left(
      \begin{array}{cc}
        #1 & #2\\
        #3 & #4
      \end{array}
    \right)
  }
}

\newcommand{\Spec}{\operatorname{Spec}}

\newcommand{\hugeslant}[2]{{\raisebox{.3em}{$#1$}\Big/\raisebox{-.3em}{$#2$}}}

  \newcommand{\bigslant}[2]{{\raisebox{.2em}{$#1$}\left/\raisebox{-.2em}{$#2$}\right.}}
\newcommand{\smallslant}[2]{{\raisebox{.1em}{$#1$}\left/\raisebox{-.1em}{$#2$}\right.}}

\newcommand{\hugediv}[2]{{\raisebox{.3em}{$#1$}\Big|\raisebox{-.3em}{$#2$}}}

% REMEMBER: \equiv - the mod congruence sing
% \cong - the triangle congruence sign

\newcommand{\dd}{\text{d}}
\newcommand{\Llra}{\Longleftrightarrow}
\newcommand{\llra}{\longleftrightarrow}
\newcommand{\lra}{\longrightarrow}
\newcommand{\Lra}{\Longrightarrow}
\newcommand{\ra}{\rightarrow}
\newcommand{\la}{\leftarrow}
\newcommand{\ua} {\uparrow}
\newcommand{\da} {\downarrow}

\newcommand{\dydx}{\frac{\dd y}{\dd x}}
\newcommand{\ddx}{\frac{\dd }{\dd x}}

\newcommand{\lap}[1]{\ensuremath\left( \mathcal{L} \left[#1\right]\right)}
\newcommand{\InvLap}[1]{\ensuremath\left( \mathcal{L}^{-1} \left[#1\right]\right)}

\newcommand{\vech}[2]{\ensuremath\left(\begin{array}{c}#1\\#2\end{array}\right)}
\newcommand{\vechhh}[3]{\ensuremath\left(\begin{array}{c}#1\\#2\\#3\end{array}\right)}
\newcommand{\twotwomat}[4]{\ensuremath\left(\begin{array}{cc}#1& #2\\#3 & #4\end{array}\right)}

\newcommand{\dbar}[1]{\ensuremath\dot{\bar{#1}}}
\newcommand{\Ree}[1]{\ensuremath \operatorname{Re}#1}

\newcommand{\Tr}{\operatorname{Tr}}
\newcommand{\sign}{\operatorname{sign}}

% bilinear form, <.|.>
\newcommand{\bFormA}[2]{\ensuremath\left\langle #1 \Big| #2 \right\rangle}
\newcommand{\smallBFormA}[2]{\ensuremath\left\langle #1 | #2 \right\rangle}

% bilinear form, [.|.]
\newcommand{\bFormB}[2]{\ensuremath\left[ #1 \Big| #2 \right]}

\newcommand{\Imm}{\ensuremath\operatorname{Im}}

\newcommand{\id}{\ensuremath\operatorname{Id}}
%\newcommand{\ker}{\ensuremath\operatorname{ker}}

\newcommand{\Sym}{\ensuremath\operatorname{Sym}}
\newcommand{\ol}{\overline}

\newcommand{\mbf}[1]{\boldsymbol{#1}}

\newcommand{\tagg} {\fbox{\fbox{\Large{\textbf{{\color{red}A}{\color{yellow}l}{\color{green}e}{\color{blue}x}}}}}}

\newcommand{\lege}[1]{\ensuremath\left\langle#1\right\rangle}

\newcommand{\abs} [1] {\ensuremath \left|#1\right|}

%\newcommand{\hat} [1] {\text{\^{$#1$}}}


%     Freddie mekanik:
\newcommand{\tder}[1]{\frac{\text{d}#1}{\text{d}t}}
\newcommand{\ud}{\mathrm{d}} % för integraler
\newcommand{\vc}[1]{\overline{\boldsymbol{#1}}}
\newcommand{\ttder}[1]{\frac{\text{d}^2#1}{\text{d}t^2}} % andra tidsderivatan
\newcommand{\dtder}[1]{\dot{#1}}


% Alex mekanik:
\newcommand{\uvc} [1] {\ensuremath\widehat{\vc{#1}}} % unit vector

\newcommand{\floor} [1] {\left\lfloor #1 \right\rfloor}

\newcommand{\legendre} [2] {\left(\frac{#1} {#2}\right)}


\title{Smutsig geometri}
\author{Alex Loiko}

\begin{document}

\maketitle

\section{Vad?}
\textit{Syntetisk geometri}, vilket betyder den geometri som grekerna höll på med,
brukar på engelska kallas för \textit{pure}. Men sedan de antika grekerna 
har man uppfunnit
algebra, trigonometri, koordinater, vektorer, komplexa tal och en massa annat som 
gör svåra geometriproblem möjliga att lösa även för oss som inte har 
läst alla 13 volymer av Elementa. Det är dessa metoder jag vill fokusera på.

Det blir ungefär följande ämnen i ungefär den här ordningen:
\begin{description}
\item[Inversion] en transformation som vänder planet ut och in runt en cirkel.
    Kan användas för att göra om komplicerade cirklar till enkla raka linjer.

\item[Projektiv geometri] %\todo{TODO: describe!}

\item[Koordinatgeometri och vektorer] 

\item[Barycentriska koordinater] - ett alternativt koordinatsystem för planet 
som använder $3$ koordinater givet $3$ startpunkter $A, B, C$. Till exempel
kommer $A$ bli $(1, 0, 0)$, $B$ är $(0, 1, 0)$ och 
den punkt $F$ som minimerar $|AF|+|BF|+|CF|$ är
\begin{eqnarray*}
\left(-\frac{a^4+a^2 \left(b^2+c^2+4 \sqrt{3} S\right)-2 \left(b^2-c^2\right)^2}{3 a^4-2 a^2 \left(3 b^2+3 c^2+2 \sqrt{3} S\right)+3 b^4-2 b^2 \left(3 c^2+2 \sqrt{3} S\right)+3 c^4-4 \sqrt{3} c^2 S},\right.\\
-\frac{-2 a^4+a^2 \left(b^2+4 c^2\right)+b^4+b^2 \left(c^2+4 \sqrt{3} S\right)-2 c^4}{3 a^4-2 a^2 \left(3 b^2+3 c^2+2 \sqrt{3} S\right)+3 b^4-2 b^2 \left(3 c^2+2 \sqrt{3} S\right)+3 c^4-4 \sqrt{3} c^2 S},\\
\left.-\frac{-2 a^4+a^2 \left(4 b^2+c^2\right)-2 b^4+b^2 c^2+c^4+4 \sqrt{3} c^2 S}{3 a^4-2 a^2 \left(3 b^2+3 c^2+2 \sqrt{3} S\right)+3 b^4-2 b^2 \left(3 c^2+2 \sqrt{3} S\right)+3 c^4-4 \sqrt{3} c^2 S}\right)
\end{eqnarray*}
\end{description}
där $S$ är triangelns area, $a,b,c$ - sidlängerna $|BC|, |AC|, |AB|$ respektive.

\section{För vem?}
För alla intresserade så klart! Rekommenderade förkunskaper:
 
\begin{description}
\item[Inversion]: Likformiga trianglar och vinklar.

\item[Projektiv geometri]: här blir det också från grunden.

\item[Koordinater och vektorer]: Säkra källor säger att de som har gått ett halvt år i ettan
    kan tillräckligt.
%\todo{Osäker på vad ni kan och vad jag vill göra;
%    har inte kommit dit i planerandet än}.

\item[Barycentriska koordinater]: Lite trigonometri och ett hum om vektorer.

\end{description}
\section{När, var och hur mycket?}
    Att meddelas, att bestämmas och upp till ett par timmar för varje delämne.
    Jag har en del lediga eftermiddagar och en del lediga helger. Det finns möjlighet att hålla det på SU eller KTH,
    annars har jag inget emot att åka in till Dagy.

\section{Läxor och kursmaterial}
    Jag tänker dela ut frivilliga övningsuppgifter. Jag hämtar material från
    \begin{description}
    \item[Geometry Unbound] av Kiran Kedlaya, \url{http://kskedlaya.org/geometryunbound/}
    \item[Geometry Revisited] av Coxeter och Greitzer. Det kan cirkulera en fil 
    av tvivelaktig laglighet på nätet. Jag tar med en döda-träd-verision till dem som
    är intresserade.
    \item[Ännu oskrivna lektionsanteckningar] av mig.
    \end{description}

\end{document}

