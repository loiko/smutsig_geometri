
\section{Inversion}
En matematisk avbildning som vänder ett plan ut och in.

Givet en cirkel med centrum $O$ och radien $r$ på planet $\Pi$, definiera en transformation $f$:

\[ f: \Pi \setminus \{ 0 \} \functo \Pi \setminus \{ 0 \} \]
där $P \in \Pi \setminus \{ 0 \}$ avbildas på $P'$,

så att $P'$ är på andra sidan cirkeln och $\abs{OP} \abs{OP'} = r^2$.

\subsection{Konstruktioner med passare och linjal}
Konstruktioner i planet med cirklar, linjer, punkter. Tillåtna operationer:

\begin{itemize}
\item Rita cirkel: givet punkter $A,B,O$, konstruera en cirkel med centrum i $O$ och radie längden av $AB$.
\item Hitta skärningspunkter
\item Dra linje genom två punkter
\end{itemize}
%(hänvisar till postulat 1--3)

\subsubsection{Exempel: flytta en vinkel}
Målet är att flytta vinkeln mellan $l_1,l_2$ till punkten $P$ på linjen $l_3$.

%Låt $A$ vara skärningspunkten mellan $l_1,l_2$.
%En cirkel $\omega_1$ dras med centrum i $A$ som går genom $P$. Detta bildar punkten $Q$ på linjen $l_3$.

Det ingår lite handviftning när skärningspunkt väljs: det finns ofta två skärningspunkter och vi väljer den som passar.

\begin{eqnarray*}
A = l_1 \cap l_3 \\
\omega_1 = cirkel(A, \abs{AP}) \\
\omega_2 = cirkel(B, \abs{AP}) \\
B = \omega_1 \cap l_1 \\
C = \omega_1 \cap l_2 \\
Q = \omega_2 \cap l_3 \\
\omega_3 = cirkel(Q, \abs{BC}) \\
R = \omega_2 \cap \omega_1 \\
l_4 = linje(P,R)
\end{eqnarray*}

\subsubsection{Uppgift: konstruera inversionen av punkten $P$}
Skriv om förhållandet $\abs{OP} \abs{OP'} = r^2$ till $\frac{\abs{OP'}}{r} = \frac{r}{\abs{OP}}$. Det kan tolkas som en likformighet.

Välj en punkt $A$ på cirkeln (inte på linjen genom $P$)\footnote{Kan konstrueras t.ex. genom att $P$ är centrum och radien $OP$}. Avståndet $AO$ är då $r$. 

Flytta sedan vinkeln $OPA$ till linjen $OA$. Bilda så vinkeln $P'$ på linjen $OP$.

Med lite algebra kan man visa att det är en korrekt punkt. Triangeln $OAP'$ är då likformig med $OPA$, då de har två gemensama vinklar. Förhållandet följer då från denna likformighet.

\subsection{Lemma: Inversion vänder på vinklar}
Låt $P$, $Q$ vara två punkter som inte ligger på linje med $O$. Antag att $P$ ligger innanför och $Q$ utanför. 

Det gäller att $\abs{OP}\abs{OP'} = \abs{OQ} \abs{OQ'}$.

Skriv om detta som ett triangelförhållande, så får man två kongruenta trianglar.
%% \begin{eqnarray*}
%% \frac{}
%% \end{eqnarray*}

\subsection{Cirklar avbildas på cirklar}

\subsubsection{Sats}
Låt $\Gamma$ vara en cirkel med centrum $O$. Punkterna $A,B,C,D$ ligger på samma cirkel som ej går genom $O$. Då kommer deras avbildningar $A',B',C',D'$ också ligga på samma cirkel.

%% TODO: visades detta?

\subsubsection{Korollarium}
Cirklar ej genom $O$ avbildas på cirklar ej genom $O$.

Bevis: Låt $\Omega$ vara en cirkel ej genom $O$. 
Låt $A,B,C$ vara olika punkter på dessa.
Beteckna med $\Omega'$ cirkeln genom $A',B',C'$.

Om $D$ är en punkt på $\Omega$, kommer $D'$ ligga på samma cirkel som $A',B',C'$ enligt tidigare sats. Men cirkeln genom tre punkter är unik, alltså $D' \in \Omega$.

Alltså avbildas $\Omega$ på $\Omega'$. Appliceras avbildningen en gång till ser man att endast punkter på $\Omega$ avbildas på $\Omega'$. 

Detaljer lämnas åt läsaren av det oskrivna kompendiet.

\subsection{Försök till bevis av satsen}
Begravdes.

\subsection{Cirklar genom origo går på linjer}
Bevis: Cirklar går på linjer. Detta för att linjer går på cirklar. Detta för att cirklar går på linjer. Det är ett cirkelbevis. -- Nicole

Bevisidé:
Låt en cirkel närma sig origo. Denna kommer bli allt större och övergår i en oändlig cirkel -- en linje -- vid inversionen.

\subsubsection{Lemma}
Vad händer om cirkeln ligger innanför inversionscirkeln? Linjen hamnar utanför inversionscirkeln.

Vad händer om cirkeln tangerar inversionscirkeln? Det blir en tangent till inversionscirkeln.

