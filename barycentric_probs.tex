\documentclass[10pt,a4paper]{article}
\usepackage[utf8]{inputenc}
\usepackage[T1]{fontenc}
\usepackage{graphicx}
\usepackage{amsmath}
\usepackage{amssymb}
\usepackage{amsthm}
\usepackage[swedish]{babel}
\usepackage[colorlinks]{hyperref}
\usepackage{url}
%\usepackage{qtree}
\usepackage{color}
\usepackage{todonotes}

% commutative diagrams
\usepackage[all,cmtip]{xy}
%\usepackage{alg}

%\usepackage{youngtab}

% Add this one at the last possible moment,
% don't forget -shell-escape (one '-'),
% never use anything non-ascii
% and just delete everything and start again
% if you get strange errors.
% Oh, and did I mention, you have to have python
% and pygments installed!
% On the other hand, the results are worth it!
% \usepackage{texments}
% \usestyle{default}

\usepackage{breqn}

\newtheorem{Theorem}{Theorem}
\newtheorem{Uppgift}{Uppgift}
\newtheorem{Lemma}{Lemma}
\newtheorem{Lösning}{Lösning}
\newtheorem{Solution}{Solution}
\newtheorem{Definition}{Definition}
\newtheorem{Example}{Example}
\newtheorem{Sats}{Sats}
\newtheorem{Exempel}{Exempel}
\newtheorem{Proposition}{Proposition}
\newtheorem{Corollary}{Corollary}
\newtheorem{Exercise}{Exercise}
\newtheorem{Claim}{Claim}

\newcommand{\bigRedBox}[1]
{
  \fbox{
    \parbox{300px}{
      \color{red}
      #1
    }
  }
}

\newcommand{\redBox}[1]
{
  \fbox{
    \color{red}
    #1
  }
}


\newcommand{\smallLine}
{\begin{center}
    \line(1,0){100}
  \end{center}
}
\newcommand{\myLine}
{\begin{center}\line(1,0){200}\end{center}}

\newcommand{\myInput}[1]
{
  \vspace{1cm}
  \begin{center}
    \textbf{File: #1}\\  
    \line(1,0){350}
  \end{center}
  \input{#1}
}

%\newcommand{\dd}{\text{d}}
\newcommand{\Lla}{\Longleftarrow}
%\newcommand{\llra}{\longleftrightarrow}
%\newcommand{\lra}{\longrightarrow}
%\newcommand{\Lra}{\Longrightarrow}


%\addtolength{\textwidth}{1in}
%\addtolength{\hoffset}{-0.5in}
%\addtolength{\textheight}{1in}
%\addtolength{\voffset}{-1in}

\newcommand{\twoMat}[4]{
  \ensuremath{
    \left(
      \begin{array}{cc}
        #1 & #2\\
        #3 & #4
      \end{array}
    \right)
  }
}

\newcommand{\Spec}{\operatorname{Spec}}

\newcommand{\hugeslant}[2]{{\raisebox{.3em}{$#1$}\Big/\raisebox{-.3em}{$#2$}}}

  \newcommand{\bigslant}[2]{{\raisebox{.2em}{$#1$}\left/\raisebox{-.2em}{$#2$}\right.}}
\newcommand{\smallslant}[2]{{\raisebox{.1em}{$#1$}\left/\raisebox{-.1em}{$#2$}\right.}}

\newcommand{\hugediv}[2]{{\raisebox{.3em}{$#1$}\Big|\raisebox{-.3em}{$#2$}}}

% REMEMBER: \equiv - the mod congruence sing
% \cong - the triangle congruence sign

\newcommand{\dd}{\text{d}}
\newcommand{\Llra}{\Longleftrightarrow}
\newcommand{\llra}{\longleftrightarrow}
\newcommand{\lra}{\longrightarrow}
\newcommand{\Lra}{\Longrightarrow}
\newcommand{\ra}{\rightarrow}
\newcommand{\la}{\leftarrow}
\newcommand{\ua} {\uparrow}
\newcommand{\da} {\downarrow}

\newcommand{\dydx}{\frac{\dd y}{\dd x}}
\newcommand{\ddx}{\frac{\dd }{\dd x}}

\newcommand{\lap}[1]{\ensuremath\left( \mathcal{L} \left[#1\right]\right)}
\newcommand{\InvLap}[1]{\ensuremath\left( \mathcal{L}^{-1} \left[#1\right]\right)}

\newcommand{\vech}[2]{\ensuremath\left(\begin{array}{c}#1\\#2\end{array}\right)}
\newcommand{\vechhh}[3]{\ensuremath\left(\begin{array}{c}#1\\#2\\#3\end{array}\right)}
\newcommand{\twotwomat}[4]{\ensuremath\left(\begin{array}{cc}#1& #2\\#3 & #4\end{array}\right)}

\newcommand{\dbar}[1]{\ensuremath\dot{\bar{#1}}}
\newcommand{\Ree}[1]{\ensuremath \operatorname{Re}#1}

\newcommand{\Tr}{\operatorname{Tr}}
\newcommand{\sign}{\operatorname{sign}}

% bilinear form, <.|.>
\newcommand{\bFormA}[2]{\ensuremath\left\langle #1 \Big| #2 \right\rangle}
\newcommand{\smallBFormA}[2]{\ensuremath\left\langle #1 | #2 \right\rangle}

% bilinear form, [.|.]
\newcommand{\bFormB}[2]{\ensuremath\left[ #1 \Big| #2 \right]}

\newcommand{\Imm}{\ensuremath\operatorname{Im}}

\newcommand{\id}{\ensuremath\operatorname{Id}}
%\newcommand{\ker}{\ensuremath\operatorname{ker}}

\newcommand{\Sym}{\ensuremath\operatorname{Sym}}
\newcommand{\ol}{\overline}

\newcommand{\mbf}[1]{\boldsymbol{#1}}

\newcommand{\tagg} {\fbox{\fbox{\Large{\textbf{{\color{red}A}{\color{yellow}l}{\color{green}e}{\color{blue}x}}}}}}

\newcommand{\lege}[1]{\ensuremath\left\langle#1\right\rangle}

\newcommand{\abs} [1] {\ensuremath \left|#1\right|}

%\newcommand{\hat} [1] {\text{\^{$#1$}}}


%     Freddie mekanik:
\newcommand{\tder}[1]{\frac{\text{d}#1}{\text{d}t}}
\newcommand{\ud}{\mathrm{d}} % för integraler
\newcommand{\vc}[1]{\overline{\boldsymbol{#1}}}
\newcommand{\ttder}[1]{\frac{\text{d}^2#1}{\text{d}t^2}} % andra tidsderivatan
\newcommand{\dtder}[1]{\dot{#1}}


% Alex mekanik:
\newcommand{\uvc} [1] {\ensuremath\widehat{\vc{#1}}} % unit vector

\newcommand{\floor} [1] {\left\lfloor #1 \right\rfloor}

\newcommand{\legendre} [2] {\left(\frac{#1} {#2}\right)}


	
\begin{document}
\section*{Barycentriska koordinater}
\begin{enumerate}


\item\textbf{Egenskaper hos barycentriska koordinater}.
	Sidorna av en (icke-degenererad) triangel $ABC$ delar in planet i $7$ delar.
	I insidan av triangeln har varje punkt $3$ positiva koordinater ($(+,+,+)$). På
	linjen $BC$ mellan $B$ och $C$ är koordinaterna $(0, +, +)$. För varje återstående område: beskriv tecknen på koordinaterna. Varje koordinat är $\in \{+,-,0\}$.

\item \textbf{Unika koordinater}.
$A,B,C$ är tre olika punkter som inte ligger på samma linje (ej kolinjära).
Visa att de barycentriska koordinaterna relativt $A,B,C$ är unika: varje koordinat
motsvarar en unik punkt, och varje punkt har en unik koordinat.
{\tiny{
        Bevisa det här först: Om $\alpha A + \beta B + \gamma C = 0$ med $\alpha + \beta + \gamma = 1$, så ligger
        $A, B, C$ på en linje. } }

\item \textbf{Oberoende av origo}. Visa att de barycentriska koordinaterna 
	baserade på en triangel $A,B,C$ i $2$ dimensioner är oberoende av valet på origo.
	(Analogt med det $1$-dimensionella fallet).

        

\item \textbf{Linjer genom samma punkt}. 
	I en triangel $ABC$ med är punkterna $D \in BC$, $E \in AC$, $F \in AB$ sådana att
	$AD, BE, CF$ skär varandra i en punkt $P$. Låt $D'$ vara speglingen av $D$ i mittpunkten av $BC$,
	$E'$ - speglingen av $E$ i mittpunkten av $AC$ och analogt för $F'$. Visa att 
	de tre nya linjerna $AE', BE', CF'$ skär varandra i en punkt $P'$. Hitta de barycentriska 
	koordinaterna för $P'$ om $P = (n, m, l)$.
	

\item \textbf{Trigonometriska Ceva}. I en triangel $ABC$ ligger punkterna
	$D$ på $BC$, $E$ på $AC$, $F$ på $AB$. Visa att $AD, BE, CF$ skär varandra
	i samma punkt om och endast om 
	\[
	\frac{\sin \angle BAD \sin \angle ACF \sin \angle CBE}
	     {\sin \angle DAC \sin \angle FCB \sin \angle EBA}
	\]
	%\textit{Hint: Sök på \textit{Ceva's theorem}}

\item \textbf{Linjer genom samma punkt}. 
	I en triangel $ABC$ med är punkterna $D \in BC$, $E \in AC$, $F \in AB$ sådana att
	$AD, BE, CF$ skär varandra i en punkt $P$. Låt $AA', BB', CC'$ vara bisektriserna 
	dragna ur $A, B, C$. Låt linjen $a$ vara speglingen av $AD$ i linjen $AA'$, $b$ och $c$ analogt.
	Visa att de tre linjerna $a,b,c$ skär varandra i en punkt. \textit{(Svårt)} Vad är koordinaterna av den
	punkten uttryckt i $P$? 

	{\tiny{Om du kör fast - sök på \textit{isogonal conjugate}}}.

\item Härled koordinaterna för ditt favorit-triangelcentrum. Om du inte har något: 
	gå in på \textsc{Encyclopedia of Triangle Centers} \url{http://faculty.evansville.edu/ck6/encyclopedia/ETC.html}
	och skaffa en! Bonuspoäng om du behöver använda ett datoralgebra-system och koordinaterna inte får
	plats på pappret.
        
        Min favorit är \textit{Fermat-Torricelli}-punkten vars koordinater är 
        \begin{eqnarray*}
        \left(-\frac{a^4+a^2 \left(b^2+c^2+4 \sqrt{3} S\right)-2 \left(b^2-c^2\right)^2}{3 a^4-2 a^2 \left(3 b^2+3 c^2+2 \sqrt{3} S\right)+3 b^4-2 b^2 \left(3 c^2+2 \sqrt{3} S\right)+3 c^4-4 \sqrt{3} c^2 S},\right.\\
        -\frac{-2 a^4+a^2 \left(b^2+4 c^2\right)+b^4+b^2 \left(c^2+4 \sqrt{3} S\right)-2 c^4}{3 a^4-2 a^2 \left(3 b^2+3 c^2+2 \sqrt{3} S\right)+3 b^4-2 b^2 \left(3 c^2+2 \sqrt{3} S\right)+3 c^4-4 \sqrt{3} c^2 S},\\
        \left.-\frac{-2 a^4+a^2 \left(4 b^2+c^2\right)-2 b^4+b^2 c^2+c^4+4 \sqrt{3} c^2 S}{3 a^4-2 a^2 \left(3 b^2+3 c^2+2 \sqrt{3} S\right)+3 b^4-2 b^2 \left(3 c^2+2 \sqrt{3} S\right)+3 c^4-4 \sqrt{3} c^2 S}\right)
        \end{eqnarray*}



\end{enumerate}

%% \section{Projektiv geometri}.
%% \item\textbf{Projektiv cirkel}.
%% 	Härled ekvationen för en cirkel i det projektiva planet. Alltså vilket samband måste $(\alpha, \beta, \gamma)$ i 
%% 	$[\alpha : \beta : \gamma]$ uppfylla för att ligga på en given cirkel?

%% \item\textbf{Dubbelförhållande}.
%% 	Låt $A,B,C,D$ vara $4$ olika punkter på en linje. Då är 
%% 	\[
%% 	\frac{AC \cdot BD}
%% 	{AD \cdot BC}
%% 	\]
%% 	punkternas dubbelförhållande. Visa \textit{(ganska krångligt)} att dubbelförhållande bevaras under projektiva transformationer
%% 	förutsatt att ingen punkt av $A,B,C,D$ skickas på oändligheten.

%% \item\textbf{Desargues sats}. Visa omvändningen till Desargues sats.

\end{document}
