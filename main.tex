\documentclass[10pt,a4paper]{article}
\usepackage[swedish]{babel} 
\usepackage[utf8]{inputenc}
\author{Freddie Agestam} 
\title{Antecktingar från ``Smutsig geometri''} 
\date{\today} 

\usepackage{amssymb}
\usepackage{amsmath}
\usepackage{style}


\begin{document}

%\newcommand{\abs}[1]{|#1|}
%\newcommand{\functo}{\rightarrow}


\maketitle

\section{Inversion}
En matematisk avbildning som vänder ett plan ut och in.

Givet en cirkel med centrum $O$ och radien $r$ på planet $\Pi$, definiera en transformation $f$:

\[ f: \Pi \setminus \{ 0 \} \functo \Pi \setminus \{ 0 \} \]
där $P \in \Pi \setminus \{ 0 \}$ avbildas på $P'$,

så att $P'$ är på andra sidan cirkeln och $\abs{OP} \abs{OP'} = r^2$.

\subsection{Konstruktioner med passare och linjal}
Konstruktioner i planet med cirklar, linjer, punkter. Tillåtna operationer:

\begin{itemize}
\item Rita cirkel: givet punkter $A,B,O$, konstruera en cirkel med centrum i $O$ och radie längden av $AB$.
\item Hitta skärningspunkter
\item Dra linje genom två punkter
\end{itemize}
%(hänvisar till postulat 1--3)

\subsubsection{Exempel: flytta en vinkel}
Målet är att flytta vinkeln mellan $l_1,l_2$ till punkten $P$ på linjen $l_3$.

%Låt $A$ vara skärningspunkten mellan $l_1,l_2$.
%En cirkel $\omega_1$ dras med centrum i $A$ som går genom $P$. Detta bildar punkten $Q$ på linjen $l_3$.

Det ingår lite handviftning när skärningspunkt väljs: det finns ofta två skärningspunkter och vi väljer den som passar.

\begin{eqnarray*}
A = l_1 \cap l_3 \\
\omega_1 = cirkel(A, \abs{AP}) \\
\omega_2 = cirkel(B, \abs{AP}) \\
B = \omega_1 \cap l_1 \\
C = \omega_1 \cap l_2 \\
Q = \omega_2 \cap l_3 \\
\omega_3 = cirkel(Q, \abs{BC}) \\
R = \omega_2 \cap \omega_1 \\
l_4 = linje(P,R)
\end{eqnarray*}

\subsubsection{Uppgift: konstruera inversionen av punkten $P$}
Skriv om förhållandet $\abs{OP} \abs{OP'} = r^2$ till $\frac{\abs{OP'}}{r} = \frac{r}{\abs{OP}}$. Det kan tolkas som en likformighet.

Välj en punkt $A$ på cirkeln (inte på linjen genom $P$)\footnote{Kan konstrueras t.ex. genom att $P$ är centrum och radien $OP$}. Avståndet $AO$ är då $r$. 

Flytta sedan vinkeln $OPA$ till linjen $OA$. Bilda så vinkeln $P'$ på linjen $OP$.

Med lite algebra kan man visa att det är en korrekt punkt. Triangeln $OAP'$ är då likformig med $OPA$, då de har två gemensama vinklar. Förhållandet följer då från denna likformighet.

\subsection{Lemma: Inversion vänder på vinklar}
Låt $P$, $Q$ vara två punkter som inte ligger på linje med $O$. Antag att $P$ ligger innanför och $Q$ utanför. 

Det gäller att $\abs{OP}\abs{OP'} = \abs{OQ} \abs{OQ'}$.

Skriv om detta som ett triangelförhållande, så får man två kongruenta trianglar.
%% \begin{eqnarray*}
%% \frac{}
%% \end{eqnarray*}

\subsection{Cirklar avbildas på cirklar}

\subsubsection{Sats}
Låt $\Gamma$ vara en cirkel med centrum $O$. Punkterna $A,B,C,D$ ligger på samma cirkel som ej går genom $O$. Då kommer deras avbildningar $A',B',C',D'$ också ligga på samma cirkel.

%% TODO: visades detta?

\subsubsection{Korollarium}
Cirklar ej genom $O$ avbildas på cirklar ej genom $O$.

Bevis: Låt $\Omega$ vara en cirkel ej genom $O$. 
Låt $A,B,C$ vara olika punkter på dessa.
Beteckna med $\Omega'$ cirkeln genom $A',B',C'$.

Om $D$ är en punkt på $\Omega$, kommer $D'$ ligga på samma cirkel som $A',B',C'$ enligt tidigare sats. Men cirkeln genom tre punkter är unik, alltså $D' \in \Omega$.

Alltså avbildas $\Omega$ på $\Omega'$. Appliceras avbildningen en gång till ser man att endast punkter på $\Omega$ avbildas på $\Omega'$. 

Detaljer lämnas åt läsaren av det oskrivna kompendiet.

\subsection{Försök till bevis av satsen}
Begravdes.

\subsection{Cirklar genom origo går på linjer}
Bevis: Cirklar går på linjer. Detta för att linjer går på cirklar. Detta för att cirklar går på linjer. Det är ett cirkelbevis. -- Nicole

Bevisidé:
Låt en cirkel närma sig origo. Denna kommer bli allt större och övergår i en oändlig cirkel -- en linje -- vid inversionen.

\subsubsection{Lemma}
Vad händer om cirkeln ligger innanför inversionscirkeln? Linjen hamnar utanför inversionscirkeln.

Vad händer om cirkeln tangerar inversionscirkeln? Det blir en tangent till inversionscirkeln.

\section{Barycentriska koordinater}
Barus = tung, massiv


\subsection{1-dimensionella barycentriska koordinater}
Ett plan, 2 dimensioner, $\R^2 = \{ (x,y) | x \in \R, y \in R \}$.

\begin{eqnarray*}
O = (0, 0) \\
A = (x_1, y_1) \\
B = (x_2, y_2) \\
\end{eqnarray*}


De 1-dimensionella barycentriska koordinater relativt $A$, $B$ är tupler $(\alpha, \beta)$ med $\alpha, \beta \in \R$, $\alpha + \beta = 1$, som syftar på punkter $\alpha A + \beta B$.

Varför hamnar alla sådana punkter på en linje? Vi tittar på ett exempel först.

Ett exempel: punkten mellan dem.

Vi \emph{definierar} en linje $AB$ som alla punkter $\{ A + \lambda (B - A) | \lambda \in \R \} \subseteq \R$.

\subsubsection{Sats: Unik representation och entydighet}
Varje punkt $P \in \overline{AB}$ har en unik $(\alpha, \beta)_{AB} = P$. Varje $(\alpha,\beta)_{AB} = P$ motsvarar unik $P \in \overline{AB}$.

% Citat: Får vi inte *minus* B här?  Vi *borde* inte få det, vi borde få rätt sak.

Uppgift: Givet $A,B$ och en punkt $P$ på en linje, samt avstånden från $P$ till $A$ resp $B$, hitta koordinaterna för $P$.

Lösning: 
\[ A + \frac{\lambda}{\mu + \lambda} (B - A) = \left( \frac{\mu}{\lambda + \mu}, \frac{\lambda}{\lambda + \mu} \right)_{AB} \]

\subsubsection{Sats: Koordinater oberoende av origo}

\begin{eqnarray*}
& \text{Låt } & O' = O + \delta = O + (\delta_x, \delta_y) \\
&& A' = A + \delta \\
&& B' = B + \delta \\
& \text{Låt } & P = (\alpha, \beta)_{AB} \\
&& P' = P - \delta \\
& \text{Då är } & P' = (\alpha, \beta)_{A',B'} 
\end{eqnarray*}

\subsection{2-dimensionella barycentriska koordinater}
Ett plan, 2 dimensioner, $\R^2 = \{ (x,y) | x \in \R, y \in R \}$.

\begin{eqnarray*}
O = (0, 0) \\
A = (x_1, y_1) \\
B = (x_2, y_2) \\
C = (x_3, y_3) \\
\end{eqnarray*}
$A, B, C$ är inte på samma linje.

De 2-dimensionella barycentriska koordinaterna relativt $A$, $B$, $C$ är tupler $(\alpha, \beta, \gamma)$ med $\alpha, \beta, \gamma \in \R$, $\alpha + \beta + \gamma = 1$, som syftar på punkter $\alpha A + \beta B + \gamma C$.

\subsubsection{Sats: Unik representation och entydighet}
Varje punkt $P \in \R^2$ har en unik $(\alpha, \beta, \gamma)_{ABC} = P$. Varje $(\alpha,\beta,\gamma)_{ABC} = P$ motsvarar unik $P$.

Uppgift: Givet $A,B$ och en punkt $P$ på en linje, samt avstånden från $P$ till $A$ resp $B$, hitta koordinaterna för $P$.


\subsubsection{Sats: Oberoende av origo}
Formulera och bevisa satsen själva.

\subsubsection{Användning: Medianernas skärningspunkt}
Medianerna i en triangel skärs i en punkt i förhållandet $1:2$.

TODO, finns bilder

\subsubsection{Cevas sats}
Triangel $ABC$. Givet punkt $P$ i triangeln, bestäm var punkten från $A$ genom $P$ skär motstående sida, dvs skärningspunkt $AP \cap BC$.

Bevis (finns bild): kalla den sökta punkten $Q = (O, \alpha, 1-\alpha)_{ABC}$. Då får vi linjen 
\begin{eqnarray*}
& \overline{AP} =& \beta (1, 0, 0) + (1, \beta)(l,m,n) \\
&& (\beta (
\end{eqnarray*}

Resten av beviset är rätt långt. Se foton.

Satsen lyder...
$A,B,C$ ligger ej på linje. Tag 
\begin{eqnarray*}
A' \in \overline{BC} \\
B' \in \overline{AC} \\
C' \in \overline{AB}
\end{eqnarray*}
Då gäller att linjerna skärs i en punkt $P$ omm
\[ \frac{AB'}{B'C} \frac{CA'}{A'B} \frac{BC'}{C'A} = 1 \]


\section{Projektiv geometri}
$x,y,z \in \R$, sådana att $(x,y,z) \neq (0,0,0)$.
Låt $[ x : y : z ] = \{ \lambda (x,y,z) | \lambda \in \R \}$.

% --------------------

En punkt $(x,y,-1) \in \Pi$ motsvarar exakt en linje $[ x : y : -1]$. En linje $[x:y:0]$

\end{document}

