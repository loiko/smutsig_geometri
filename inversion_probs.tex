\documentclass[10pt,a4paper]{article}
\usepackage[utf8]{inputenc}
\usepackage[T1]{fontenc}
\usepackage{graphicx}
\usepackage{amsmath}
\usepackage{amssymb}
\usepackage{amsthm}
\usepackage[swedish]{babel}
\usepackage[colorlinks]{hyperref}
\usepackage{url}
%\usepackage{qtree}
\usepackage{color}
\usepackage{todonotes}

% commutative diagrams
\usepackage[all,cmtip]{xy}
%\usepackage{alg}

%\usepackage{youngtab}

% Add this one at the last possible moment,
% don't forget -shell-escape (one '-'),
% never use anything non-ascii
% and just delete everything and start again
% if you get strange errors.
% Oh, and did I mention, you have to have python
% and pygments installed!
% On the other hand, the results are worth it!
% \usepackage{texments}
% \usestyle{default}

\usepackage{breqn}

\newtheorem{Theorem}{Theorem}
\newtheorem{Uppgift}{Uppgift}
\newtheorem{Lemma}{Lemma}
\newtheorem{Lösning}{Lösning}
\newtheorem{Solution}{Solution}
\newtheorem{Definition}{Definition}
\newtheorem{Example}{Example}
\newtheorem{Sats}{Sats}
\newtheorem{Exempel}{Exempel}
\newtheorem{Proposition}{Proposition}
\newtheorem{Corollary}{Corollary}
\newtheorem{Exercise}{Exercise}
\newtheorem{Claim}{Claim}

\newcommand{\bigRedBox}[1]
{
  \fbox{
    \parbox{300px}{
      \color{red}
      #1
    }
  }
}

\newcommand{\redBox}[1]
{
  \fbox{
    \color{red}
    #1
  }
}


\newcommand{\smallLine}
{\begin{center}
    \line(1,0){100}
  \end{center}
}
\newcommand{\myLine}
{\begin{center}\line(1,0){200}\end{center}}

\newcommand{\myInput}[1]
{
  \vspace{1cm}
  \begin{center}
    \textbf{File: #1}\\  
    \line(1,0){350}
  \end{center}
  \input{#1}
}

%\newcommand{\dd}{\text{d}}
\newcommand{\Lla}{\Longleftarrow}
%\newcommand{\llra}{\longleftrightarrow}
%\newcommand{\lra}{\longrightarrow}
%\newcommand{\Lra}{\Longrightarrow}


%\addtolength{\textwidth}{1in}
%\addtolength{\hoffset}{-0.5in}
%\addtolength{\textheight}{1in}
%\addtolength{\voffset}{-1in}

\newcommand{\twoMat}[4]{
  \ensuremath{
    \left(
      \begin{array}{cc}
        #1 & #2\\
        #3 & #4
      \end{array}
    \right)
  }
}

\newcommand{\Spec}{\operatorname{Spec}}

\newcommand{\hugeslant}[2]{{\raisebox{.3em}{$#1$}\Big/\raisebox{-.3em}{$#2$}}}

  \newcommand{\bigslant}[2]{{\raisebox{.2em}{$#1$}\left/\raisebox{-.2em}{$#2$}\right.}}
\newcommand{\smallslant}[2]{{\raisebox{.1em}{$#1$}\left/\raisebox{-.1em}{$#2$}\right.}}

\newcommand{\hugediv}[2]{{\raisebox{.3em}{$#1$}\Big|\raisebox{-.3em}{$#2$}}}

% REMEMBER: \equiv - the mod congruence sing
% \cong - the triangle congruence sign

\newcommand{\dd}{\text{d}}
\newcommand{\Llra}{\Longleftrightarrow}
\newcommand{\llra}{\longleftrightarrow}
\newcommand{\lra}{\longrightarrow}
\newcommand{\Lra}{\Longrightarrow}
\newcommand{\ra}{\rightarrow}
\newcommand{\la}{\leftarrow}
\newcommand{\ua} {\uparrow}
\newcommand{\da} {\downarrow}

\newcommand{\dydx}{\frac{\dd y}{\dd x}}
\newcommand{\ddx}{\frac{\dd }{\dd x}}

\newcommand{\lap}[1]{\ensuremath\left( \mathcal{L} \left[#1\right]\right)}
\newcommand{\InvLap}[1]{\ensuremath\left( \mathcal{L}^{-1} \left[#1\right]\right)}

\newcommand{\vech}[2]{\ensuremath\left(\begin{array}{c}#1\\#2\end{array}\right)}
\newcommand{\vechhh}[3]{\ensuremath\left(\begin{array}{c}#1\\#2\\#3\end{array}\right)}
\newcommand{\twotwomat}[4]{\ensuremath\left(\begin{array}{cc}#1& #2\\#3 & #4\end{array}\right)}

\newcommand{\dbar}[1]{\ensuremath\dot{\bar{#1}}}
\newcommand{\Ree}[1]{\ensuremath \operatorname{Re}#1}

\newcommand{\Tr}{\operatorname{Tr}}
\newcommand{\sign}{\operatorname{sign}}

% bilinear form, <.|.>
\newcommand{\bFormA}[2]{\ensuremath\left\langle #1 \Big| #2 \right\rangle}
\newcommand{\smallBFormA}[2]{\ensuremath\left\langle #1 | #2 \right\rangle}

% bilinear form, [.|.]
\newcommand{\bFormB}[2]{\ensuremath\left[ #1 \Big| #2 \right]}

\newcommand{\Imm}{\ensuremath\operatorname{Im}}

\newcommand{\id}{\ensuremath\operatorname{Id}}
%\newcommand{\ker}{\ensuremath\operatorname{ker}}

\newcommand{\Sym}{\ensuremath\operatorname{Sym}}
\newcommand{\ol}{\overline}

\newcommand{\mbf}[1]{\boldsymbol{#1}}

\newcommand{\tagg} {\fbox{\fbox{\Large{\textbf{{\color{red}A}{\color{yellow}l}{\color{green}e}{\color{blue}x}}}}}}

\newcommand{\lege}[1]{\ensuremath\left\langle#1\right\rangle}

\newcommand{\abs} [1] {\ensuremath \left|#1\right|}

%\newcommand{\hat} [1] {\text{\^{$#1$}}}


%     Freddie mekanik:
\newcommand{\tder}[1]{\frac{\text{d}#1}{\text{d}t}}
\newcommand{\ud}{\mathrm{d}} % för integraler
\newcommand{\vc}[1]{\overline{\boldsymbol{#1}}}
\newcommand{\ttder}[1]{\frac{\text{d}^2#1}{\text{d}t^2}} % andra tidsderivatan
\newcommand{\dtder}[1]{\dot{#1}}


% Alex mekanik:
\newcommand{\uvc} [1] {\ensuremath\widehat{\vc{#1}}} % unit vector

\newcommand{\floor} [1] {\left\lfloor #1 \right\rfloor}

\newcommand{\legendre} [2] {\left(\frac{#1} {#2}\right)}


	
\begin{document}
\begin{enumerate}

\item \textbf{Egenskaper av inversion} Beskriv vad som händer när en triangel inverseras i \begin{enumerate}
	\item dess incirkel\footnote{Den inskrivna cirkeln som tangerar alla tre sidorna}
	\item dess omcirkel\footnote{Den omskrivna cirkeln, dvs den cirkel som går genom de tre hörnen}
	\end{enumerate}

\item \textbf{Egenskaper av inversion} $P,Q$ är godtyckliga punkter. $O$ är centrum för en cirkel $\omega$ med radie $r$. Låt $P',Q'$ vara 
	de inverterade avbildningarna av $P,Q$. Finn avståndet $|PQ|$ uttryckt i $r, |OP|, |OQ|, |PQ|$.

\item \textbf{USAMO, lite förenklat} En fyrhörning $ABCD$ har $AC \perp BD$\footnote{vinkeln mellan $AC$ och $BD$ är $90^\circ$}. Låt 
	$E=AC\cap BD$\footnote{skärningspunkten}. Låt $P,Q,R,S$ vara projektionerna av $E$ på sidorna $AB, BC, CD, AD$.\footnote{D.v.s. 
	dra en höjd ner från $E$ till varje sida och låt $P,Q,R,S$ vara skärningspunkter mellan sidorna och höjderna}.
	Visa att $PQRS$ är cyklisk.

\item \textbf{Steiners porism}
	Två cirklar $\Gamma_1, \Gamma_2$ och ett tal $n$ 
	är sådana att de inte skär varandra, att $\Gamma_1$ ligger inuti $\Gamma_2$ och 
	att det finns $n$ cirklar $C_1, C_2, \ldots C_n$ med 
	\begin{itemize}
	\item $C_1$ tangerar $C_{n}, C_{2}, \Gamma_1, \Gamma_2$
	\item $C_2$ tangerar $C_{1}, C_{3}, \Gamma_1, \Gamma_2$
	\item $\ldots \ \vdots \ \ldots\ \vdots\ \ldots$
	\item $C_n$ tangerar $C_{n-1}, C_{1}, \Gamma_1, \Gamma_2$
	\end{itemize}

	\smallLine

	Låt $\omega_1$ vara en ny cirkel (inte ev an $C_1,\ldots, C_n$) som tangerar $\Gamma_1, \Gamma_2$.
	Konstruera $\omega_2$ så att den tangerar $\omega_1, \Gamma_1, \Gamma_2$ (det går alltid).
	Fortsätt med att konstruera $\omega_3$ mellan $\omega_2, \Gamma_1, \Gamma_2$.
	Visa att när du kommer till $\omega_n$, så kommer den \textbf{alltid} att tangera $\omega_1$, oavsett var den började.
	D.v.s. att man kan ta de $n$ cirklarna $C_1,\ldots,C_n$ och låta dem glida runt kontinuerligt mellan $\Gamma_1, \Gamma_2$.

	\smallLine
	Använd följande påstående utan bevis:
	\textit{
	Om $\Gamma_1, \Gamma_2$ är två cirklar som inte skär varandra, så finns det en annan cirkel och en inversion i den
	som avbildar $\Gamma_1, \Gamma_2$ på två koncentriska\footnote{Två cirklar runt samma centrum} cirklar.
	}

\end{enumerate}
\end{document}